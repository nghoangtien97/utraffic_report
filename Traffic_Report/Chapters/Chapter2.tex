% Chapter 2

\chapter{Các nghiên cứu liên quan} % Main chapter title

\label{Chapter2} % For referencing the chapter elsewhere, use \ref{Chapter1} 

Trước tình hình giao thông ngày càng đông đúc và dày đặc tại các thành phố lớn, trong đó có thành phố Hồ Chí Minh, nhiều biện pháp, nghiên cứu cũng được đề ra để giải quyết vấn đề nhức nhối này \cite{FSPPM}. Tuy nhiên, phần lớn các báo cáo khoa học hay nghiên cứu đều tập trung vào cách thu thập dữ liệu và đưa ra dự đoán 

Phần lớn các báo cáo khoa học hay nghiên cứu chủ yếu tập trung vào cách thu thập dữ liệu (data collection method) về luồng giao thông (traffic flow) nhằm phân tích và đưa ra dự đoán bằng những thiết bị cảm biến, radar, sóng âm hay nhận diện bằng hình ảnh, video hay thậm chí ước lượng thủ công (manual count) [10] [11]. Các phương pháp này đều cần một hệ thống cơ sở hạ tầng hiện đại, tiên tiến để thu thập dữ liệu và phân tích sau đó đưa ra kết quả nhằm mục đích nghiên cứu lâu dài.

Sở giao thông thành phố Hồ Chí Minh cũng đã có đưa ra ứng dụng [12] [13] cho phép người dân có thể xem tình trạng giao thông từ các camera đặt tại các ngã 4, ngã 3 tại thời điểm hiện tại, thông báo về hiện trạng giao thông tại các tuyến đường qua các màn hình lớn trên đường. Hệ thống này còn thu thập dữ liệu từ GPS của các tuyến xe buýt để đưa ra những đánh giá về tốc độ có thể di chuyển được trên những tuyến đường. Đại học Bách Khoa thành phố Hồ Chí Minh cũng có một nhóm nghiên cứu hiện thực một trang web thông báo tình trạng giao thông từ GPS của các tuyến xe bus [13] tương tự như hệ thống trên. Tuy nhiên những hệ thống này đều dựa vào các thiết bị định vị GPS hay các hệ thống cảm biến cố định (fixed sensor). Từ năm 2014 đến giữa năm 2018, số lượng người sử dụng thiết bị điện thoại thông minh gia tăng rất nhanh (hình
2.1) đã đưa ra một hướng thu thập dữ thiệu khác từ chính những chiếc điện thoại thông minh này.

Sự phát triển của các thiết bị di động đã tạo điều kiện cho một hướng thu thập dữ liệu khác đó là từ cộng đồng, hay còn gọi là thu thập dữ liệu cộng đồng (crowdsourcing). Có thể hiểu crowsourcing như là một tổ chức, một công ty hay một cá nhân nào đó nhằm thu thập dữ liệu, thông tin để hoàn thành một nhiệm vụ nào đó từ cộng đồng [14] [15]. Sau khi nhận được yêu cầu, cộng đồng này sẽ đưa ra giải pháp, cung cấp dữ liệu cho người yêu cầu để hoàn thành mục đích của họ. Đổi lại, cộng đồng này sẽ nhận được những phần thưởng tương xứng như tiền, danh tiếng (reputation) trong cộng đồng, sự tôn trọng từ những người trong cộng đồng, quyền xem các dữ liệu hay thông tin do những người khác từ cộng đồng chia sẻ và sự thỏa mãn khi cảm thấy mình có đóng góp giúp ích cho xã hội.

Việc sử dụng nguồn lực cộng đồng (crowdsourcing) để thu thập dữ liệu hiện đang được các tổ chức, công ty [16] tích hợp vào các hệ thống hỏi đáp, truy xuất thông tin như Wikipedia, Yahoo! Answer, Youtube, Stackoverflow..., và trong các thảm họa thiên tai trong khoảng 10 năm trở lại đây [17] [18] [19] [20] [21] [22]. Các trang mạng xã hội như Facebook, Twitter... hiện nay để có một tính năng thông báo về sự an toàn của bản thân trong các tình huống xảy ra thảm họa thiên tai. Vào năm 2010, trong trận động đất ở Haiti, hệ thống Ushahidi đã hoạt động rất hiệu quả trong việc thu thập dữ liệu, phân tích và đưa ra những chỉ dẫn trực tiếp cho các tình nguyện viên để sơ tán, di tản người dân Haiti về khu vực an toàn [23] [24].

Tuy việc áp dụng những cảnh báo từ cộng đồng nhằm thu thập các thông tin về tình trạng giao thông đã xuất hiện nhưng việc xác thực dữ liệu (data validity) vẫn chưa thật sự được chú trọng mặc dù đây là một phần vô cùng quan trọng nhằm xác định độ chính xác của dữ liệu cũng như để phân tích và đưa ra những dự đoán. Do đó, nhóm nghiên cứu đã đưa ra giải pháp nhằm giải quyết bài toàn này. Bên cạnh đó, các dữ liệu sau khi được chia sẻ bởi cộng đồng cũng được nghiên cứu để thực hiện việc gom cụm (cluster) nhằm đem đến do cộng đồng một cái nhìn đơn giản, dễ hình dung hơn về tình trạng giao thông trên bản đồ của thành phố Hồ Chí Minh. Việc nghiên cứu này đã được hiện thực bằng một phiên bản ứng dụng di động trên nền tảng Android và iOS để tiếp tục tiến hành
việc kiểm tra tính hiệu quả của quá trình nghiên cứu.