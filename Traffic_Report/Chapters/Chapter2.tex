% Chapter 2

\chapter{Các nghiên cứu liên quan} % Main chapter title

\label{Chapter2} % For referencing the chapter elsewhere, use \ref{Chapter1} 

Trước tình hình giao thông ngày càng đông đúc, phức tạp tại các thành phố lớn, trong đó có thành phố Hồ Chí Minh. Nhiều biện pháp, nghiên cứu cũng được đề ra để giải quyết vấn đề nhức nhối này \cite{FSPPM}. Cũng đã có rất nhiều báo cáo khoa học và nghiên cứu đề xuất giải pháp giải quyết bài toàn này. Tuy nhiên, phần lớn các báo cáo khoa học hay nghiên cứu đều tập trung vào cách thu thập dữ liệu về luồng giao thông bằng những thiết bị cảm biến, nhận diện bằng hình ảnh, video và đưa ra dự đoán \cite{VEGV} \cite{RTD}. Các phương pháp này đều cần một hệ thống cơ sở hạ tầng hiện đại, tiên tiến để thu thập dữ liệu và phân tích sau đó đưa ra kết quả nhằm mục đích nghiên cứu lâu dài.

Ở Việt Nam, cụ thể là thành phố Hồ Chí Minh, sở giao thông thành phố cũng đã đưa ra ứng dụng \cite{VOVAPP} cho phép người dân xem xét dòng giao thông hiện tại hoặc xem các hình ảnh trực tiếp của tuyến đường có gắn camera. Đồng thời, nhóm nghiên cứu Intelligent Transport System tại Đại học Bách Khoa thành phố Hồ Chí Minh cũng thực hiện một trang web \cite{HCMUT} thông báo tình trạng giao thông từ GPS của các tuyến xe bus tương tự các hệ thống trên đem lại các kết quả khả quan.

Bên cạnh đó, còn có một hướng đi khác là thu thập dữ liệu từ cộng đồng (crowdsourcing), đã được nhiều tổ chức, công ty \cite{16} sử dụng hiệu quả khi tích hợp vào các hệ thống của mình. Điển hình là hệ thống Ushahidi hoạt động dựa trên mô hình crowdsourcing, trong trận động đất ở Haiti năm 2010, đã hoạt động rất hiệu quả trong việc thu thập dữ liệu, phân tích và đưa ra những chỉ dẫn trực tiếp cho các tình nguyện viên để sơ tán, di tản người dân Haiti về khu vực an toàn \cite{23} \cite{24}.

Kế thừa từ những nghiên cứu, kết quả đã đạt được ở trên. Nhóm nghiên cứu mong muốn đưa ra hệ thống kết hợp dữ liệu lấy được từ các hệ thống giao thông sử dụng dữ liệu GPS, đồng thời xây dựng nguồn dữ liệu crowdsourcing bằng ứng dụng di động và các phương pháp khuyến khích chia sẻ dữ liệu cũng như xác thực tính đúng đắn của dữ liệu để đưa ra những dự đoán tình hình giao thông một cách hoàn thiện và chính xác nhất. 