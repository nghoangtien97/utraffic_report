% Chapter 7

\chapter{Đánh giá} % Main chapter title

\label{Chapter7}

\section{Kết quả đạt được}
Nhóm đã tìm hiểu các đề tài liên quan, các nguồn dữ liệu như HERE Real-time Traffic, Google Maps, Smart BK Traffic, crowdsourcing...

Nắm được cách thu thập dữ liệu thông qua API cung cấp từ Smart BK Trafic, liên kết với các đề tài liên quan để thu thập dữ liệu từ cộng đồng.

Từ các dữ liệu đã thu thập được, tiến hành thiết kế cơ sở dữ liệu.

Bước đầu xây dựng ứng dụng web với chức năng cơ bản là hiển thị tình trạng giao thông, thông qua việc tô màu tương ứng với tốc độ lưu thông trên bản đồ.

\section{Hạn chế}
Một số nguồn dữ liệu có độ tin cậy không cao. Bên cạnh đó Google Maps không cung cấp dữ liệu giao thông gây ra khó khăn trong khi thu thập dữ liệu.

Do mới ở kỳ thực hiện đề cương luận văn, cũng như thiếu kinh phí nên web chỉ mới chạy trên localhost.

Việc nghiên cứu cũng chỉ ở giai đoạn thu thập dữ liệu, chưa khai thác về Data Mining để thực hiện dự đoán dựa trên dữ liệu đã có.