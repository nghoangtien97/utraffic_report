% Chapter 5

\chapter{Đánh giá và định hướng trong tương lai} % Main chapter title

\label{Chapter5}
\section{Đánh giá}
\subsection{Kết quả đạt được}
Nhóm đã tìm hiểu các đề tài liên quan, các nguồn dữ liệu như HERE Real-time Traffic, Google Maps, Smart BK Traffic, crowdsourcing...

Nắm được cách thu thập dữ liệu thông qua API cung cấp từ Smart BK Trafic, liên kết với các đề tài liên quan để thu thập dữ liệu từ cộng đồng.

Từ các dữ liệu đã thu thập được, tiến hành thiết kế cơ sở dữ liệu.

Bước đầu xây dựng ứng dụng web với chức năng cơ bản là hiển thị tình trạng giao thông, thông qua việc tô màu tương ứng với tốc độ lưu thông trên bản đồ.

\subsection{Hạn chế}
Một số nguồn dữ liệu có độ tin cậy không cao. Bên cạnh đó Google Maps không cung cấp dữ liệu giao thông gây ra khó khăn trong khi thu thập dữ liệu.

Do mới ở kỳ thực hiện đề cương luận văn, cũng như thiếu kinh phí nên web chỉ mới chạy trên localhost.

Việc nghiên cứu cũng chỉ ở giai đoạn thu thập dữ liệu, chưa khai thác về Data Mining để thực hiện dự đoán dựa trên dữ liệu đã có.

\subsection{Kết luận}
Với sự phát triển phức tạp của giao thông ở thành phố Hồ Chí Minh, tình hình ùn tắc giao thông ngày càng trở nên nghiêm trọng. Việc thu thập, phân tích và dự đoán tình trạng giao thông vì vậy mà trở nên rất cần thiết.

Qua giai đoạn đầu của quá trình nghiên cứu đề tài, nhóm đã thu được nhiều kết quả khả quan và một số hạn chế nhất định. Những kết quả đạt được và kinh nghiệm rút ra là bước chuẩn bị để thực hiện đề tài vào thời gian tới.

\section{Định hướng trong tương lai}
Ở giai đoạn tiếp theo, nhóm sẽ tiếp tục phát triển đề tài theo cá mục tiêu:
\begin{itemize}
    \item Tiếp tục tìm hiểu và thu thập dữ liệu để có kết quả khách quan và chính xác hơn.
    \item Nghiên cứu các kỹ thuật Data Mining để tối ưu hoá độ chính xác khi dự đoán.
    \item Từ các dữ liệu đã thu thập được, thực hiện khai phá dữ liệu để dự đoán tình trạng giao thông, nhất là ở những đoạn đường chưa có dữ liệu trực tiếp.
    \item Hoàn thiện ứng dụng web và xây dựng thêm một số chức năng hỗ trợ người dùng.
\end{itemize}